\documentclass{article}

\title{Solving Rubik's Cube with Lego Mindstorms}
\date{01-01-1901}
\author{Will Garside}

\usepackage[a4paper,left=2.54cm,right=2.54cm, top=2.54cm, bottom=2.54cm]{geometry}
\usepackage{booktabs}
\usepackage{tabularx}
\usepackage{multirow}
\usepackage{gensymb}
\usepackage{csquotes}
\usepackage{xcolor}

\begin{document}
		
	\pagenumbering{gobble}
	\maketitle
	\newpage
	
	\pagenumbering{roman}
	\section*{Signed Declaration}
	\addcontentsline{toc}{section}{Signed Declaration}
	All sentences or passages quoted in this report from other people's work have been specifically acknowledged by clear cross-referencing to author, work and page(s). Any illustrations which are not the work of the author of this report have been used with the explicit permission of the originator and are specifically acknowledged. I understand that failure to do this amounts to plagiarism and will be considered grounds for failure in this project and the degree examination as a whole.
	\\
	\\Name: Will Garside
	\\Signature:
	\\Date: 

	\newpage
	\section*{Abstract}
	\addcontentsline{toc}{section}{Abstract}
	This should be two or three short paragraphs (100-150 words total), summarising the report. A suggested flow is background, project aims, and achievements to date. It should not simply be a restatement of the original project outline
	
	\newpage
	\section*{Acknowledgements}
	\addcontentsline{toc}{section}{Acknowledgements}
	Thanks to my parents, who raised me since I was a boy. And Rik van Grol, who raised me afterwards.
	
	\newpage
	\section*{Abbreviations, Definitions and Notations}
	\addcontentsline{toc}{section}{Abbreviations, Definitions and Notations}
	\subsection*{Basic Definitions}
	\addcontentsline{toc}{subsection}{Basic Definitions}
	\begin{table}[htbp]
		\def\arraystretch{1.25}
		\begin{tabular}{p{0.25\textwidth}m{0.65\textwidth}}
			\toprule
			\textbf{Abbreviation/Word} & \textbf{Definition} \\
			\midrule
			Rubik's Cube 		& 	\multirow{ 2}{*}{A standard 3x3x3 Rubik's Cube\textsuperscript{\ref{cube definition}}.} \\
			Cube 				& 	\\
%			Face 				& 	One of the size faces on a Cube. Denoted by position (LRFBUD) or colourl.\\
			Slice				&	A central layer between two faces. Usually referenced by the faces it spans. \\
			Quarter Turn		&	A clockwise rotation of a face or slice by 90\degree \\
			Half Turn			&	A clockwise rotation of a face or slice by 180\degree \\
			Quarter Turn Metric	&	When counting moves, a Quarter Turn is counted as a single move and a half is two moves. \\
			Half Turn Metric\textsuperscript{\ref{metric definition}}	&	Quarter turns and half turns are both counted as single moves. \\
			Cubie				&	One of the twenty-six smaller cubes that make up a Cube. \\
			Goal State			&	All the cubies on a given face match the colour of the centre cubie. i.e. a solved Cube. \\
			Position			&	A Cube's state (mixed or solved). \\
			Valid Position		&	A position that can be achieved with a real-world Cube without dismantling it. \\
			\multirow{ 2}{*}{Move Sequence}		&	A series of moves performed consecutively. \\
			&	e.g. $F\;D\;F'\;D2\;L'\;B'\;U\;L\;D\;R\;U\;L'\;F'\;U\;L\;U2$ \\
			Solve (Sequence) 	&	A move sequence which leads to the goal state. \\
			Depth $n$			&	A Cube which has been moved $n$ times away from the goal state. \\
			\bottomrule
		\end{tabular}
	\end{table}
	\subsection*{Notation}
	\addcontentsline{toc}{subsection}{Notation}
	\begin{table}[htbp]
		\def\arraystretch{1.25}
		\begin{tabular}{p{0.25\textwidth}m{0.2\textwidth}m{0.425\textwidth}}
			\toprule
			\textbf{Symbol Notation} & \textbf{Meaning} \\
			\midrule
			$L$	&	Left			&	\multirow{ 9}{*}{\parbox{0.425\textwidth}{This notation is used to show a quarter turn of a face. It can have a single quote or a '2' appended to show a counter-clockwise quarter turn or a half turn respectively.}} \\
			$R$	&	Right			&	\\
			$F$	&	Front			&	\\
			$B$	&	Back			&	\\
			$U$	&	Up (Top)		&	\\
			$D$	&	Down (Bottom)	&	\\
			$X$ &	\multirow{ 3}{*}{\parbox{0.2\textwidth}{Clockwise 90\degree{} rotation of a Cube about the relevant axis.}} & \\
			$Y$ & & \\
			$Z$ & & \\
			\bottomrule
		\end{tabular}
	\end{table}
	% Content for footnote is here because it wouldn't display from table %
	\textcolor{white}{
		\footnote{\label{cube definition}This Dissertation is only dealing with 3x3x3 Rubik’s Cubes, and any discussion of alternative dimensions will be explicitly stated.}
		\footnote{\label{metric definition}For this Dissertation the Half Turn Metric is used unless explicitly stated.}		
	}



	\newpage
	\tableofcontents
	
	\newpage
	\pagenumbering{arabic}
	\section{Introduction}
    \subsection{Background}
    \paragraph{}
    In 1974, Ern\"{o} Rubik was struggling to create a cube with independently moving parts which remain together, regardless of how much they moved. His first attempts made use of elastic, which broke and rendered the cube unusable. Rubik persevered in his attempts to hold the blocks (now called \enquote{cubies}) together - eventually concluding that the best way was to have the cubes hold themselves together. He called this design \enquote{The Magic Cube}, and it would go on to be one of the world's best-selling puzzles \cite{Waxman2014b}. It was later re-branded to \enquote{\textit{Rubik's Cube}} to overcome an oversight involving patenting and copyrighting  the design.
    
    \paragraph{}
    In an unpublished manuscript \cite{Rubik1986}, Rubik described first randomising his new cube, \blockquote{It was wonderful to see how, after only a few turns, the colors became mixed, apparently in random fashion. Like after a nice walk when you have seen many lovely sights you decide to go home, after a while I decided it was time to go home, let us put the cubes back in order. And it was at that moment that I came face to face with the Big Challenge: What is the way home?}
    
    \paragraph{}
    It took Rubik over a month to solve this first cube - he knew intuitively that there must be a method to solving the cube, but lacked the finer methodology \cite{RubiksCube}. Since Rubik devised the first method, hobbyists and mathematicians alike have been immersed in solving the Cube as quickly and efficiently as possible. Whilst many solutions are markedly successful when it comes to optimisation, others only better them in quirkiness or internet fame \cite{Chan2016}.
    
    \paragraph{}
    There is one method which is mere speculation, despite having been proved mathematically: God's Algorithm. God's Algorithm states that an omniscient being would always make the most efficient moves and that they would be able to solve a Cube from any given position in a certain number of moves or less. This is referred to as God's Number, and was finally proved to be twenty in 2010 by a group of 4 researchers \cite{Rokicki2010b}.
    
    \begin{appendix}
    	\newpage  
    	\listoffigures
    	\listoftables
    	\newpage
    	\bibliographystyle{ieeetr}
    	\bibliography{library}
    \end{appendix}
    
\end{document}